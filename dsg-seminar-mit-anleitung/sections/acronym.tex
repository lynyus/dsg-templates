%
\section{Nutzung des Abk�rzungsverzeichnisses}
%
Zur Einbindung des Abk�rzungsverzeichnisses wird das \texttt{acronym}--Package\footnote{\url{http://www.ctan.org/tex-archive/macros/latex/contrib/acronym}} verwendet.
Der Eintrag aller verwendeten Abk�rzungen erfolgt in der Datei \texttt{abbreviations.tex}, die bspw. den folgenden Inhalt hat:
%
\begin{verbatim}
	\begin{acronym}[LSPI]
	 \acro{DSG}{Distributed Systems Group}
	 \acro{LSPI}{Lehrstuhl f�r Praktische Informatik}
	\end{acronym}
\end{verbatim}
%
Die geklammerte Abk�rzung \texttt{[LSPI]} sollte dabei durch die l�ngste vorhandene Abk�rzung ersetzt werden.
Eine Abk�rzung wird mit dem Befehl \texttt{\textbackslash acro} innerhalb der \texttt{acronym}--Umgebung definiert.
Standardm��ig werden nur die im Text tats�chlich verwendeten Abk�rzungen auch im Verzeichnis ausgegeben.
Um eine Abk�rzung innerhalb eines Textes einzuf�gen gen�gt der folgende Ausdruck:

\texttt{\textbackslash ac\{}\textit{<acronym>}\texttt{\}}

Bei der ersten Verwendung innerhalb des Textes bewirkt die Anweisung, dass der vollst�ndige Name gefolgt von der Abk�rzung in Klammern ausgegeben wird.
Bei jeder weiteren Verwendung wird lediglich die Kurzform ausgegeben. Die Anweisung \texttt{\textbackslash ac\{DSG\}} erzeugt also zun�chst die Langform "`\ac{DSG}"'.
Anschlie�end nur noch die Abk�rzung "`\ac{DSG}"'.