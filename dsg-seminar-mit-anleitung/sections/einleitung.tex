%
\section{Einleitung}
%
Zur Verfassung von Seminar-, Bachelor- und Masterarbeiten bietet die Distributed Systems Group
neben den Word-Vorlagen auch Vorlagen für Latex. In diesem Dokument soll
kurz gezeigt, wie die einzelnen Umgebungen und Befehle der Vorlage sinnvoll zur Erstellung
von Arbeiten genutzt werden können. Es erfolgt aber keine allgemeine Anleitung zur
Erstellung von Dokumenten mit \LaTeX. Als Einstiegsliteratur, die aber für die allgemeine
Arbeit mit \LaTeX~durchaus ausreichend ist, eignet sich das Buch von Kopka \cite{kop02}. Eines
der älteren Werke des Autors reicht ebenfalls aus. 
Eine sehr gute Übersicht über alle wichtigen Szenarien gibt das Wikibook "`\LaTeX"' \cite{lat12}.
Weitere hilfreiche Lektüre findet sich
unter \cite{jue00}, \cite{kue11}, \cite{jue95}, \cite{erb09} und allen verfügbaren Suchmaschinen.

Die Erstellung eines \LaTeX-Dokumentes erfolgt z.B. über die Anweisung \texttt{pdflatex seminar} oder mit einer 
entsprechenden IDE, wie \textit{TeXnicCenter}\footnote{\url{http://www.texniccenter.org}} oder \textit{TeXlipse}\footnote{\url{http://texlipse.sourceforge.net}}.
Hier empfiehlt sich auch das Zusammenspiel mit SumatraPDF\footnote{\url{http://blog.kowalczyk.info/software/sumatrapdf/free-pdf-reader-de.html}}, das eine Vorwärts- und Rückwärtssuche in den Dokumenten ermöglicht. 