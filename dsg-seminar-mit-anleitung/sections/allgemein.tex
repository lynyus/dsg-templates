%
\section{Genereller Aufbau}
%
Die zentrale Datei der Vorlage ist die Datei \texttt{seminar.tex}, die vom Autor für jeweilige Arbeit
anzupassen ist. In dieser Datei können die Einstellungen für die Titelseite vorgenommen
werden. Des Weiteren müssen in dieser Datei die einzelnen Kapitel der Arbeit, sowie die
\textit{.bib}-Datei mit den \textit{bibtex}-Einträgen eingebunden werden.
%
\subsection{Erstellung einer Titelseite}
%
Die Erstellung der Titelseite erfolgt über den Befehl \texttt{\textbackslash maketitle}. Die Anzahl der Parameter,
mit der dieser Befehl aufgerufen wird, hängt davon ab, ob eine Seminararbeit oder
eine Bachelor-/Masterarbeit erstellt werden soll.
Zur Erstellung einer Titelseite für eine Seminararbeit muss der Befehl \texttt{\textbackslash maketitle} mit
sechs Parametern aufgerufen werden.
%
\begin{verbatim}
\maketitle{Thema des Seminars}{Thema der Ausarbeitung}%
{Autor}{Betreuer}{Semester}{Bachelor|Master}
\end{verbatim}
%
Die einzelnen Paramter sind selbsterklärend. Falls die Seminararbeit von mehreren Autoren
verfasst wird, sind diese durch \texttt{,\textbackslash\textbackslash} zu trennen. Bei drei Autoren ist also z.B. die
Zeichenkette \texttt{Hans Meier,\textbackslash\textbackslash\ Peter Müller und\textbackslash\textbackslash\ Hans Müller} als dritten Parameter
einzusetzen. Als Semesterangabe sollte ein Kürzel wie \texttt{SoSe13} übergeben werden.

Um die Titelseite einer Abschlussarbeit zu erstellen, ist der Befehl \texttt{\textbackslash maketitle} mit drei
Parametern aufzurufen.
%
\begin{verbatim}
\maketitle{Bachelor-/Masterarbeit}{Studiengang}{Thema der Arbeit}%
{Autor}{Abgabedatum}
\end{verbatim}
%
Auch in diesem Fall sind die Parameter selbsterklärend.
%
\subsection{Einbindung der einzelnen Kapitel}
%
Um die Übersichtlichkeit der erstellten Arbeit zu gewährleisten, ist es sinnvoll, für jedes
Kapitel eine einzelne \textit{.tex}-Datei zu erstellen. 
Damit unter Verwendung dieser Dateien ein einzelnes Dokument erstellt wird, sind die einzelnen Kapitel in der Datei \texttt{seminar.tex}
einzubinden. Dies erfolgt über den Befehl \texttt{\textbackslash input}. Soll z.B. ein Kapitel eingefügt werden,
dass in der Datei \texttt{kapitel-1.tex} enthalten ist, so geschieht dies durch die folgende
Anweisung.

\texttt{\textbackslash input\{kapitel-1\}}

Diese Anweisung ist an der in der Datei \texttt{seminar.tex} markierten Stelle einzufügen. 
%
\subsection{Einbindung des Literaturverzeichnisses}
%
Die einzelnen Literaturquellen sind in einer \textit{.bib}--Datei aufzuführen. Die \textit{bibtex}--Datei \texttt{example.bib} enthält einige Beispiele, wie
verschiedenen Quellen wie Buch, Artikel usw. in dieser Datei aufzuführen sind. Jeder der
Einträge benötigt zur Referenzierung ein eindeutiges Label, das sich aus den ersten drei
Buchstaben des Nachnamen des Autors und den letzten beiden Ziffern der Jahreszahl der
Veröffentlichung ergeben. Sind mehrere Autoren bei einer Literaturquelle aufgeführt, so
ergibt sich der Label aus den Anfangsbuchstaben der Nachnamen der ersten drei Autoren
und den letzten beiden Ziffern der Jahreszahl der Veröffentlichung. Falls ein Autor in einem
Jahr mehrere Werke veröffentlich hat, so sind die weiteren Label mit kleinen Buchstaben,
beginnend bei \textit{b}, zu ergänzen.

Eine \textit{.bib}-Datei ist mit Hilfe der Anweisung \texttt{\textbackslash bibliography} in der Datei \texttt{seminar.tex} einzubinden.
Die hier verwendete Datei \texttt{references.bib} wurde also durch durch folgende Anweisung
in dieses Dokument eingebunden.

\texttt{\textbackslash bibliography\{references\}}